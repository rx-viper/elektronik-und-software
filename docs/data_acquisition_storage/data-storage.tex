\documentclass[12pt, a4paper]{scrreprt}

\renewcommand{\familydefault}{\sfdefault}

\begin{document}

\subsubsection{Sensor data storage}
The initial idea was to store the sensor data with timestamps, but this has a few disadvantages.
\begin{itemize}
	\item Time resolution should be greater than $\frac{1}{50}s$, so we have to use milliseconds.
	\item Flight duration is 600s, but test may last longer.
	\item As a consequence 16bit timestamps are too short, we have to use 32bit timestamps (Our software framework \textit{xpcc} can handle both).
	\item A 32bit timestamp is bigger than the actual sensor data value, so it would increase the bandwidth and storage capacity by factor two to five.
	\item We don't have to store the time because we continuously sample all sensors.
\end{itemize}

\subsubsection{Solution}
\begin{itemize}
	\item Put multiple values from one unique sensor in a \textit{packet}, which also stores information about the sensor, its type and a sequence number.
	\item The packets can be used as buffers before sensor values are written to memory and transmitted.
\end{itemize}

\subsubsection{Sensor data packet}
\begin{tabular}{|c|c|c|c|}
	\hline 
	\multicolumn{3}{|c|}{\textbf{Header}} & \textbf{Payload} \\ 
	\hline 
	Sensor type & Sensor ID & Sequence Number  & Sensor values \\ 
	\hline 
	$2\,bit$ & $6\,bit$ & $4\,Byte$ & $123\,Byte$ \\ 
	\hline 
	\multicolumn{4}{|c|}{\textbf{Total packet size: 128 Byte}} \\ 
	\hline 
\end{tabular} 

\end{document}          
